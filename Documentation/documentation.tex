\documentclass[10pt,a4paper]{article}
\usepackage[T1]{fontenc}
\usepackage[utf8]{inputenc}
\usepackage{graphicx}
\usepackage{tabularx}
\usepackage{helvet}
\usepackage{hyperref}
\usepackage[a4paper,margin=1in]{geometry}
\usepackage[polish]{babel}
\usepackage{float}

\renewcommand\familydefault{\sfdefault}

\begin{document}
\begin{titlepage}
	\centering
	{\Large Wydział Matematyki i Nauk Informacyjnych Politechniki Warszawskiej \par}
	\vspace{1cm}
	\includegraphics[width=0.2\textwidth]{logo.png} \par
	\vspace{5cm}
	{\LARGE Tytuł projektu \par}
	\vspace{0.5cm}
	{\Large Autorzy \par}
	\vspace{1.5cm}
	{\Large Nr wersji dokumentu \par}
	\vspace{1.5cm}
	{\Large \today \par}
\end{titlepage}
Lista zmian w dokumencie:
\begin{table}[H]
\def\arraystretch{1.5}
\begin{tabularx}{\textwidth}{|l|l|X|l|}
	\hline
	\textbf{Data} & \textbf{Autor} & \textbf{Opis zmian} & \textbf{Wersja} \\
	\hline
	11.10.2016 & Bartłomiej Dach & Stworzenie szablonu \LaTeX & 1.0 \\
	\hline
\end{tabularx}
\end{table}

\tableofcontents
\newpage

\section{Specyfikacja}

\subsection{Opis biznesowy}
% Punkt obowiązkowy.
%
% W niniejszym punkcie należy zawrzeć, jaki jest cel tworzonego systemu/aplikacji, jakie funkcje ma on
% spełniać i kto jest jego użytkownikiem.

\subsection{Wymagania biznesowe}
% Punkt obowiązkowy.
%
% W niniejszym punkcie należy zawrzeć, wymagania tworzonego systemu/aplikacji, dozwolone formy:
% diagramy przypadków użycia wraz z opisem, historie użytkownika (user stories) wraz z opisem,
% pomocniczo: scenariusze użycia.

\subsubsection*{Przypadki użycia}
\subsubsection*{User stories}

\subsection{Wymagania niefunkcjonalne}
% Punkt obowiązkowy.
%
% Rozdział zawierać powinien listę wymagań niefunkcjonalnych oraz ich opis najlepiej w formie tabeli. W
% uchwyceniu wymagań funkcjonalnych przydatna jest klasyfikacja wymagań FURPS+.
% FURPS jest akronimem od:
% - Functionality – wymaganie funkcjonalne, przedstawione powyżej
% - Usability – wymagania związane z użytecznością i ergonomią systemu
% - Reliability – wymaganie związane z niezawodnością i dostępnością systemu
% - Performance – wymagania związane z wydajnością systemu
% - Supportability – wymagania związane z utrzymaniem systemu
% + ma za zadanie przypominać o szczególnych wymaganiach związanych np. z koniecznością użycia
% konkretnej technologii, wzorców projektowych, metodyki zarządzania projektem, bezpieczeństwem
% danych, których istnienie zależy od specyfiki branży i systemu
% Zadaniem prowadzącego jest zdefiniowanie jakie obszary wymagań niefunkcjonalnych poza URPS
% powinny być wyspecyfikowane.

\subsection{Harmonogram projektu}
% Punkt obowiązkowy.
%
% Rozdział powinien zawierać harmonogram projektu obejmujący co najmniej kamienie
% milowe.
% Celem tworzenia harmonogramu jest oczywiście zaplanowanie prac nad projektem, co w
% szczególności pozwala m.in. na identyfikację następstwa zadań w projekcie. Ważnym
% efektem jest często stwierdzenie, iż np. projekt wymaga szybkiego uruchomienia pierwszych
% zadań i/lub ich częściowo równoległej realizacji, gdyż bez podjęcia tych działań zakończenie
% projektu w planowanym terminie będzie trudne bądź niemożliwe. Harmonogram projektu
% zawarty w tym punkcie musi definiować co najmniej kamienie milowe, aczkolwiek z przyczyn
% powyżej zalecane jest również wydzielenie zadań o czasie trwania rzędu jednego - trzech
% tygodni lub dłuższym (stopień szczegółowości zależny od czasu trwania całego projektu).
% Wśród kamieni milowych należy wyróżnić co najmniej poniższe:
% - zakończenie analizy kluczowych wymagań odzwierciedlone w punktach 1.1, 1.2 i 1.3
%   niniejszego  dokumentu.  Wymagania  powinny  być  aprobowane  przez  wszystkich
%   interesariuszy (tu: wykonawców i osoby oceniające projekt)
% - zakończenie i uzgodnienie z prowadzącym specyfikacji (rozdział 1)
% - uzyskanie stabilnej architektury rozwiązania (opisanej w punkcie 1.5 i potwierdzonej np.
%   wstępną implementacją)
% - gotowość aplikacji stworzonej w ramach projektu do użycia
% - zakończenie dokumentacji (rozdział 2 i 3) i naniesienie zmian wynikających z testów w
%   aplikacji
% Uwaga: praca nad analizą wymagań nie wyklucza podejmowania równolegle prac nad architekturą,
% czy też wstępnej implementacji.
% 
% Opis powinien mieć postać tabeli z listą zadań i kamieni milowych, ich terminami
% rozpoczęcia i zakończenia oraz opcjonalnie diagramu Gantta, który przedstawia te
% informacje oraz zależności pomiędzy zadaniami w formie graficznej.

\subsection{Architektura rozwiązania}
% Punkt obowiązkowy.
% 
% Rozdział powinien zawierać schemat i opis architektury tworzonego z rozbiciem na poszczególne
% komponenty wchodzące w jej skład. W przypadku bardziej rozbudowanych systemów architekturę
% można opisać w formie ogólnej (pierwszy schemat) i w kolejnym kroku uszczegółowić (w kolejnym
% schemacie(ach)) co zapewni wymaganą przejrzystość rysunków. Należy również zamieścić
% uzasadnienie wybranej architektury.

\section{Dokumentacja końcowa (powykonawcza)}

\subsection{Wymagania systemowe}
% Punkt obowiązkowy.
%
% Rozdział powinien zawierać wymagania systemowe, wymagane oprogramowanie zewnętrzne
% (RDBMS, etc.)

\subsection{Biblioteki wraz z określeniem licencji}
% Punkt obowiązkowy.
%
% Rozdział powinien zawierać listę użytych bibliotek i komponentów firm trzecich wraz z ich licencjami.

\subsection{Instrukcja instalacji}
% Punkt obowiązkowy.
%
% Niniejszy rozdział powinien kompletną instrukcję instalacji systemu/aplikacji umożliwiająca osobie
% oceniającej implementację danego rozwiązania. Uwaga: instrukcja powinna być dostoswana do
% instalacji na czystym systemie operacyjnym.

\subsection{Instrukcja uruchomienia}
% Punkt obowiązkowy.
% 
% Niniejszy rozdział powinien zawierać kompletną instrukcję uruchomienia systemu/aplikacji
% umożliwiającą osobie oceniającej weryfikację poprawności działania systemu.

\subsection{Instrukcja użycia}
% Punkt obowiązkowy.
%
% Niniejszy rozdział powinien zawierać kompletną instrukcję użycia (manual) systemu/aplikacji
% umożliwiająca osobie oceniającej weryfikację poprawności działania systemu.

\subsection{Instrukcja utrzymania}
% Punkt obowiązkowy.
%
% Niniejszy rozdział powinien zawierać kompletną instrukcję utrzymania systemu obejmującą procedury
% włączenia, wyłączenia systemu oraz procedury backup/restore.

\subsection{Raport odstępstw od specyfikacji wymagań}
% Punkt obowiązkowy.

\subsection{Dokumentacja usług Web Services}
% Punkt obowiązkowy
%
% Niniejszy rozdział powinien w przypadku, gdy system udostępnia publiczne usługi web services
% powinien zawierać dokumentację usług zamieszczone w formie opisowej lub np. wg specyfikacji
% swagger.

\section{Dokumentacja końcowa (powykonawcza) -- punkty wymagane przez prowadzącego zajęcia}

\subsection{Pseudokod}

\subsection{Diagramy sekwencji}
% Punkt obowiązkowy
%
% W przypadku projektów półsemestralnych lub dłuższych: w przypadku, gdy system składa się z
% komponentów rozproszonych należy dołączyć diagram(y) sekwencji. Rozdział zawierać powinien
% zawierać diagramy sekwencji opisujące komunikację pomiędzy systemami lub komponentami
% systemu. Należy w nim uwzględnić wszystkie przepływy komunikatów pomiędzy komponentami
% systemu lub systemami.

\subsection{Model danych}
% W przypadku projektów półsemestralnych lub dłuższych, gdy system składuje dane, należy opisać
% model danych. Model danych powinien być wyrażony przez diagram entity relationship w przypadku
% relacyjnej bazy danych. Zalecane jest wówczas opisanie znaczenia poszczególnych relacji, jak
% również czytelne oznaczenie rodzaju relacji (np. jeden do wielu) oraz kluczy głównych i kluczy obcych.
% W przypadku wykorzystania platform nierelacyjnych np. platform NoSQL lub składowania danych w
% plikach Apache Hadoop należy przedstawić opis konwencji zapisu danych. Jest to szczególnie istotne,
% gdy model danych jest prowadzony w trybie tzw. schema-on-read tzn. nie jest egzekwowany przez
% platformę składowania danych, a zależy wyłącznie od konwencji stosowanej przez aplikację np.
% konwencji nazewnictwa kolumn, treści wpisów w formacie JSON lub formatu sekwencji plików
% tworzonych w systemie plików HDFS.

\subsection{Scenariusz testów akceptacyjnych}
% W przypadku projektów półsemestralnych np. realizowanych w ramach projektu zespołowego.
% Rozdział powinien zawierać scenariusze testów akceptacyjnych nawiązujących do wymagań
% funkcjonalnych i niefunkcjonalnych systemu.

\subsection{Raport z przeprowadzonych testów}
% W przypadku projektów półsemestralnych np. realizowanych w ramach projektu zespołowego.
% Rozdział powinien zawierać scenariusze testów akceptacyjnych nawiązujących do wymagań
% funkcjonalnych i niefunkcjonalnych systemu i wyniki ich przeprowadzenia.

\section{Lista użytych skrótów}

\renewcommand*{\refname}{\vspace*{-2em}}
\section{Bibliografia}
\begin{thebibliography}{99}
\bibitem{lamport94}
  Leslie Lamport,
  \emph{\LaTeX: a document preparation system},
  Addison Wesley, Massachusetts,
  2nd edition,
  1994.
\end{thebibliography}
\end{document}